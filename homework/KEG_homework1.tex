%Input the preamble from the tex_tools directory. 
\documentclass[10pt, letterpaper]{article}
\usepackage[margin=0.75in]{geometry}
\usepackage{color}
\usepackage{graphicx}
\usepackage{epstopdf}
\usepackage{microtype}
\usepackage{amsmath}
\usepackage{listings}
\usepackage{fancyhdr}
\usepackage{amssymb}
\usepackage{verbatim}
\pagestyle{fancy}
	\lhead{\vspace*{0.25in} \sc BE/Bi 103}
	\rhead{ Emily Blythe, Griffin Chure, Kyu Hyun Lee - \today}
	\renewcommand{\headrulewidth}{0.4pt}






\begin{document}
\section{Homework 1}
\noindent {\small This homework is the communal work of Emily Blythe,
Griffin Chure, and Kyu Hyun Lee. While all answers were written collectively,
answers authored primarily by one person are denoted by their initials (EC,
GC, or KL). The code used to generate the analysis and figures shown can be
found in a python script titled \texttt{KEGscript\_homework1.py}.}

\subsection{Your goals (20pts)}
\begin{itemize}
\item[\textbf{GC})] In the era of 'big data biology' where high-throughput
	methods generate data at dizzying rates, it is my sense that a
	biologist who knows only biology is severely limited in how they can
	interpret the deluge of information. Having a solid and practical
	understanding of the mathematics of scientific logic not only aids in
	the processing of data but can provide valuable insight into how
	experiments should be designed. My goal for this course is to
	learn the mathematical tenets of data analysis as well as the
	computational tools to put said tenets to use. While learning the
	language is important, I would rather walk away from this class
	knowing how to interpret data and design experiments than how
	to use Python to do data analysis. Specifically, I would like to learn
	how extract quantitative data from seemingly qualitative images (such
	as fluorescent images of bacteria). I would also like to learn how to
	write scripts that can track particle movement over both long and
	short time scales. 

\item[\textbf{EB})] As I have begun to collect data in my graduate research, I
	have realized that (1) I am likely using inefficient methods to handle
the data, making the analysis more confusing and time-consuming, and (2)
sometimes I have doubts as to how to best handle my data. Therefore, I have
two related goals for this course. First, I am hoping to learn better methods
for analyzing my data, both to save me time and to make the process as
streamlined and simple as possible. I have no prior experience using python or
any other language, so learning how to code is a big goal for me this
semester. Even if a python-based method is not the most applicable tool for my
specific data in research, learning to code will still be a valuable process.
Second, my goal for the course is to learn how to properly handle the analysis
of biological data-- which mathematical tools are available and when to apply
those tools to data. My background in mathematics that are applicable to
biological situations is a bit shaky, so I want to address this deficiency. It
seems as though this course will emphasize discussion about the best way to
analyze data with no "correct" answer, and I think that participating in these
discussions will further my understanding of appropriate data analysis.
	
\item[\textbf{KL})] My goal in this class is to be exposed to some ideas about how to process
data. There is no one ``right'' way to analyze data, but for many common
problems there are conceptual tools that have been developed and are widely
accepted as appropriate by the scientific community. In this course I hope to
be introduced to these tools and get a feel for how to use them effectively
through hands-on interactions with real data sets. 
\end{itemize}
\newpage
\subsection{Plotting functions with Python (30 pts)}
\begin{itemize}
	\item[a)]\textbf{Exponential Decay with Background - (GC)} From a
		mathematical standpoint, it is obvious that plotting the
		exponential decay function with background noise on a
		logarithmic range will provide the most information. One of
		the most important parameters describing the shape of the
		decay is the half-life, in this case denoted $\lambda$.
		Plotting the logarithm of this function with respect to $x$
		makes the half-life a parameter that can be easily measured.
		Figure 1 shows the effect of plotting the function on
		different scales.	
		\begin{center}
			\includegraphics[width=0.8\textwidth]{../figs/KEG_exp_multiplot.pdf}
		\end{center}
		{\small \textbf{Figure 1.} The exponential decay function with
		background signal plotted on various scalings. It is
		obvious that plotting on the $\log y$ scaling produces the
		most useful result. This file is saved as
		\texttt{KEG\_exp\_multiplot.pdf}.}\\

		Varying the parameters $a$, $b$, and $\lambda$ reveals what
		parameters are important in determining the shape of the
		function (Figure 2). As expected, the background signal $a$
		merely controls the linearity of the exponential decay on the
		$\log(y)$ scale. A consequence of
		taking the log of the whole function is that that the
		background signal, $a$, becomes logarithmic as well.  In the
		limit where $x \rightarrow \infty$, the value of $\log(a)$
		will become more apparent as the exponential factor shrinks,
		changing the shape of the curve.  As $a$ gets larger, it
		begins to drown out the effects of the exponential at smaller
		$x$ values. Varying $b$ merely changes the starting point of
		the decay. Varying $\lambda$ carries the largest influence on
		the decay. This term, the half-life, determines the rate at
		which the exponential falls. With large values of $\lambda$,
		the decay drastically slows whereas smaller values of
		$\lambda$ greatly increase the rate.

		\begin{center}
			\includegraphics[width=0.8\textwidth]{../figs/KEG_exp_var.pdf}
		\end{center}
		{\small \textbf{Figure 2.} Varying the parameters $a$, $b$,
		and $\lambda$ of the exponential decay function and background
		signal. It can easily be seen that $\lambda$ defines the rate
		of decay, $b$ defines the starting point of the decay, and $a$
		controls how long a decay is measurable in the presence of
	noise.}
		
	\newpage	

	\item[b)]\textbf{Cauchy Distribution - (EB)} There is no real reason
		to plot the Cauchy distribution on any manner of logarithmic
		scaling as there are no exponentiated variables. This can
		clearly be seen in Figure 3. Plotting in this way allows
		us to easily visualize the important parameters $\alpha$ and
		$\beta$ -- creating a smooth curve -- whereas these parameters
		are not obvious when plotted on the other scales.  
	
		\begin{center}
	
		\includegraphics[width=0.8\textwidth]{../figs/KEG_cauchy_multiplot.pdf}
	\end{center}	
	{\small \textbf{Figure 3.} Plotting the Cauchy distribution on
	multiple scales. Plotting on a linear scale conveys the most
	information on the effects of $\alpha$ and $\beta$. This figure is
	saved with the filename \texttt{KEG\_cauchy\_multiplot.pdf}.}\\


As shown in Figure 4, varying the
parameter $\beta$ causes a change in the shape of the curve, with a smaller
$\beta$ 
causing a more narrow function and a larger $\beta$ causing a wider function
($\beta$ represents the half width at half maximal height in the Cauchy
distribution). Changing the parameter $\alpha$ only serves to translate the
function along the $x$-axis, as shown in the figure. This result stems from the
fact that in the equation $\alpha$ only affects the value of $x$ (this changes the
center of the distribution).

\begin{center}
	\includegraphics[width=0.65\textwidth]{../figs/KEG_cauchy_var.pdf}
\end{center}
	{\small \textbf{Figure 4.} Variations of $\alpha$ and $\beta$ and their respective effects on
	the curve shape. $\beta$ defines with width of the peak and $\alpha$
	defines the position of the peak along the $x$ axis. This figure is
	saved with the filename \texttt{KEG\_cauchy\_var.pdf}.}
	
\newpage

\item[c)]\textbf{Hill Function - (KL)} As demonstrated in Figure 5, the best way to visualize the Hill
function is to plot the function on the log $x$ scale. Plotting in this way
allows us to easily visualize the important parameters $k$ (value of $x$ when
$y$ =0.5; interpretation differs depending on data type)  and $\alpha$ (shows how
sensitive $y$ is to change in $x$; again interpretation varies). This scale allows
an easy readout of both of these constants as it explicitly shows the point at
which $y$ = 0.5 and the function becomes more sharply tuned as $\alpha$ changes. It
is preferable to the log log plot because it preserves the idea that $k$ occurs
when $y$ = 0.5 rather than morphing it to the intercept. It is better than the
linear plot because plotting $x$ on log scale allows easy visualization over
many orders of magnitude. It is better than the $\log(y)$ plot because
plotting $y$ on log scale makes visualization of $y$ = 0.5 difficult.


\begin{center}
	\includegraphics[width=0.8\textwidth]{../figs/KEG_hill_multiplot.pdf}
\end{center}
	{\small \textbf{Figure 5.} Plotting the Hill function on multiple scales. Plotting on a log $x$
	scale displays the values and physical meaning of the $x$ and $k$
	parameters in an obvious way. This figure is saved as
	\texttt{KEG\_hill\_multiplot.pdf}.}

\begin{center}
	\includegraphics[width=0.65\textwidth]{../figs/KEG_hill_var.pdf}
\end{center}
{\small \textbf{Figure 6.} The effect of varying $k$ and $\alpha$ on the Hill
	function. It becomes obvious that $k$ functions as a scaling factor of
$x$, determining its placement along the $x$ axis. This file is saved as
\texttt{KEG\_hill\_var.pdf}.}



\end{itemize}

\subsection{Microtubule catastrophes I (50 pts) - EB, GC, KL}
\begin{itemize}
	\item[a)] Please see code in \texttt{KEGscript\_homework\_1.py}
		beginning at line 
		\texttt{470} to see the loading and renaming of tables in the
		\texttt{.csv} files using \texttt{pandas}.
	\item[b)] In our analysis, we had several options for displaying 
		the data regarding normalization, number of bins, and data
		display. We generated a histogram in which the data for
		the labeled and unlabeled tublin were normalized, allowing us
		to plot the frequency of microtubule collapse as a function of
		time (Figure 7).

		\begin{center}
			\includegraphics[width=0.8\textwidth]{../figs/KEG_catastrophe_histogram.pdf}
		\end{center}
			{\small \textbf{Figure 7.} Histogram of microtubule
			catastrophe events binned along time to
			catastrophe. Green and gray bars represent frequency
			of catastrophe event for GFP labeled and unlabeled
			tubulin respectively. Data sets were normalized to
			produce a measure of catastrophe frequency as opposed to a
			measure of catastrophe observations. It is evident from the plot that
			labeling the tubulin does not effect the dynamics of
		microtubule catastrophe. This file is saved as
		\texttt{KEG\_catastrophe\_histogram.pdf}.}\\


		In order to better compare the labeled and unlabeled tubulin,
		we chose to plot the two data sets on the same histogram. This
		also meant that we needed to normalize the data in some way.
		The ‘normed’ parameter normalizes the data set such that the
		area under the bars is equal to one, but we found this
		presentation of the data confusing (the $y$-axis was unclear).
		Therefore, we instead weighted each data point proportionally
		to the size of the data set so that the sum of the heights of
		the bars is equal to one. To decide how many bins to use,
		we applied the square root rule in which bin number is
		determined by $\sqrt{n}$ where $n$ is the number of data
		points. However, the two data sets
		had different sizes. We chose to use the binning size of the
		larger data set because our data is very dense around 250s. Using a
		larger number of bins (narrower bins) gives better resolution
		and precision in a dense area. For the style of the histogram,
		we used a bar histogram instead of a step histogram because
		the two data sets have great overlap, making a step histogram
		difficult to interpret.

	\item[c)] When we switched to a cumulative histogram from the histogram in part b, we
found a bar histogram difficult to interpret and visually displeasing. To solve these issues, we chose to
represent the data as points instead of bars. However, with only points we
found the plot to be slightly misleading (i.e. it looked like a scatter plot
rather than a histogram), so we added the bars back in with light colors to
emphasize that the plot is a histogram and not just a plot of raw data
points.

\begin{center}
	\includegraphics[width=0.8\textwidth]{../figs/KEG_catastrophe_cumulative_histogram.pdf}
\end{center}
	{\small \textbf{Figure 8.} Cumulative histogram of microtubule
	catastrophe with respect to time. Circles indicate the number of
	catastrophe events up that the indicated time of observation. The bars
	represent the same information but in the form of bars. Green and gray
	represent GFP labeled and unlabeled tubulin respectively. It is
	obvious that labeling of the tubulin does not affect the dynamics of
microtubule catastrophe. This file is saved as
\texttt{KEG\_catastrophe\_cumulative\_histogram.pdf}.}

\newpage
	\item[d)] We plotted the cumulative histogram with no binning (Fig.
		9). To plot the data without binning, we first sorted the data
		(and removed the NaN data). We then used the
		\texttt{np.arange} function to create arrays of evenly spaced
		points with the same size as our data sets, scaled by the size
		of the data sets. By plotting the sorted data vs. the array,
		we were able to show the cumulative distribution as a function
		of time. This plot shows that there is no skewing of the data
		by our choice of binning in the previous two plots. 

		\begin{center}
			\includegraphics[width=0.8\textwidth]{../figs/KEG_cumulative_dist.pdf}
		\end{center}
			{\small \textbf{Figure 9.} Cumulative histogram with
			no binning of
			microtubule catastrophe with respect to time. Green
			and gray circles represent total number of catastrophe
			events of GFP labeled and unlabeled tubulin. As with
			the previous two figures, it is obvious that labeling
		does not alter the behavior of the microtubule instability.
		This file is saved as \texttt{KEG\_cumulative\_dist.pdf}.}


 
\end{itemize}




\end{document}
